\documentclass{article}
\usepackage{hyperref}
\usepackage{listings}

\title{Terminal Gaming Setup}
\author{Corey Scheinfeld}

\begin{document}
	\pagenumbering{gobble}
	\maketitle
	\newpage
	\pagenumbering{arabic}

	\section{Installation}
		\paragraph{To install the package on your own computer:}
			\begin{enumerate}
				\item Download and install the latest version of Java at \url{https://java.com/en/download/}.
				\item Go to \url{https://github.com/corey-scheinfeld/_Games}
				\item Download _Games-master.zip or, if you have Git installed, clone the project with:
					\begin{lstlisting}
  git clone https://github.com/corey-scheinfeld/_Games.git
					\end{lstlisting}
				\item Open the _Games-master.zip
				\item Open the Terminal app.
				\item Change directories to the 'Games' folder. '_Games-master' is the parent of the 'Games' project folder for the zip file, '_Games' is the parent if the project was cloned. 
					\begin{lstlisting}
  cd \Path\to\folder
					\end{lstlisting}
					If you are unsure of the exact path, type 'cd ', drag the folder from your Finder window to the Terminal window, and then hit enter.
  		\end{enumerate}

\section{Playing}
				\paragraph{Battleship:}
					\begin{enumerate}
								\item Start Game by typing the following into Terminal:
									\begin{lstlisting}
javac Battleship.java
java Battleship
									\end{lstlisting}
								\item Finish game by typing control C

					\end{enumerate}
				\paragraph{ConnectFour:}
					\begin{enumerate}
								\item Start Game by typing the following into Terminal:
									\begin{lstlisting}
javac ConnectFour.java
java ConnectFour
									\end{lstlisting}
								\item Finish game by typing control C

					\end{enumerate}


\end{document}
